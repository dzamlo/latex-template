\chapter{Exemples}

\dots sur github \fnurl{https://github.com/dzamlo/svd/blob/3ca6b8808ea2606b5673ff4edfec63fd7c9c8c93/src/bit_range.rs} \dots

\dots dans le listing \ref{lst:demo1} \dots

\dots dans le listing \ref{lst:demo2} \dots

\dots dans le listing \ref{lst:struct_bit_range} \dots

\dots dans la PEP8\cite{PEP8} \dots

\includesource[pos=b, mintedoptions={firstline=6, lastline=10}, label=lst:demo1]{rust}{test-minted.rs}
\includesource[pos=H, mintedoptions={firstline=6, lastline=10}, label=lst:demo2, caption={la légende}]{rust}{test-minted.rs}
\includesource[pos=H, label=lst:demo3]{text}{test_minted2.txt}
\includesource[long, mintedoptions={breakbytoken=false}]{rust}{test-minted.rs}


% on peut aussi être plus explicite:
\begin{listing}
    \inputminted[firstline=6, lastline=10]{rust}{test-minted.rs}
    \attachcaption{test-minted.rs}{une 1\iere légende}
    \label{lst:struct_bit_range}
\end{listing}

\begin{longlisting}
	\inputminted[breakbytoken=false, highlightlines={1, 3-4}]{rust}{test-minted.rs}
	\attachcaption{test-minted.rs}{une 2\ieme légende}
\end{longlisting}




